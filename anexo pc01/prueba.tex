\documentclass[a4paper,12pt]{article}
\usepackage{fixltx2e}
\usepackage{times}
\usepackage{changepage}
\usepackage[spanish]{babel}
\usepackage{anyfontsize}
\usepackage[utf8]{inputenc}
 \usepackage{graphicx}
\usepackage{anysize} % Soporte para el comando \marginsize
\marginsize{30mm}{30mm}{30mm}{30mm}
\newenvironment{extracto}{\begin{adjustwidth}{10mm}{10mm}}{\end{adjustwidth}}
\renewcommand{\baselinestretch}{1.5}
\usepackage{float}
\floatstyle{plaintop}
\restylefloat{table}

\title{Anexo Documento PFC-01}

\author{José Ángel Parada Jiménez\\ Descripción y características del proyecto}
\date{}


\begin{document}
 \maketitle 

La idea que presento en este proyecto es la construcción de una aplicación web para la gestión de un centro óptico en el cual se podrá gestionar el centro de una forma eficaz, sencilla y sin errores de acuerdo a las exigencias y requerimientos de esta actividad comercial, poniendo en práctica así los conocimientos adquiridos durante la formación universitaria.\\
Lo que se pretende con esta propuesta es desarrollar una aplicación web, para poder implantar un sistema en la que los usuarios trabajen de manera cómoda para hacer más llevadera su labor diaria.  En la aplicación existirán varios perfiles de usuarios que podrán trabajar simultáneamente y que harán distintas funcionalidades dependiendo del rol que tuviesen. La aplicación llevará el control de los empleados, clientes, proveedores, productos, operaciones, citas e informes generando opcionalmente distintos tipos de documentos. Además, la aplicación tendrá que ser segura, funcionar en cualquier navegador, tener buen aspecto visual y tener tiempos de espera cortos.
 
\end{document}