% ------------------------------------------------------------------------------
% Este fichero es parte de la plantilla LaTeX para la realización de Proyectos
% Final de Grado, protegido bajo los términos de la licencia GFDL.
% Para más información, la licencia completa viene incluida en el
% fichero fdl-1.3.tex

% Copyright (C) 2012 SPI-FM. Universidad de Cádiz
% ------------------------------------------------------------------------------

\section{Motivación}
La elección de un proyecto fin de carrera es muy importante, ya que con él se puede mejorar conocimientos adquiridos y aprender
nuevos lenguajes de programación.\\
La motivación de este proyecto, es poner en práctica los conocimientos académicos adquiridos durante la formación universitaria en un proyecto real, y así, profundizar en aspectos del desarrollo web no tratados en el plan de estudios actual.


\section{Descripción del sistema actual}
Se concertaron varias tomas de contacto con el cliente para saber sus necesidades. En las primeras nos dimos cuenta de que 
verdaderamente su sistema presentaba carencias de funcionalidades, errores, problemas de usabilidad por lo que el cliente nos explicó cuales eran sus necesidades.
El sistema actual con el que trabajan es un software de escritorio, trabajaban desde el sistema operativo Windows, 
presenta un aspecto muy poco amigable, carencias de funcionalidades y sin copias de seguridad.

\section{Objetivos y alcance del proyecto}
Los usuarios de la aplicación deberán poder gestionar el centro óptico eficientemente y sin errores además de hacerlo cómodamente para hacer más llevadera su labor laboral diaria. Se debe intentar construir una aplicación que sea muy sencilla de utilizar ya que los usuarios del sistema
pueden tener conocimientos informáticos limitados. La aplicación será alojada en un servidor web y podría ser usada en varios puestos de trabajo simultáneamente.

\section{Organización del documento}

La memoria presenta los siguientes contenidos:\\

\begin{enumerate}

\item Introducción:

En este apartado desarrollaremos una pequeña introducción al desarrollo del proyecto describiendo apartados como motivación, descripción del sistema actual y objetos y alcance del proyecto.

\item Planificación: 

En este apartado se describe la metodología usada para la realización del proyecto y la planificación describiendo el tiempo empleado en cada fase.

\item Análisis:

En esta sección se presenta el catálogo de requisitos del sistema de información. Para ello se detallarán los actores del sistema, los requisitos funcionales, los requisitos de información, los requisitos no funcionales, las reglas de negocio y las diferentes alternativas
tecnológicas.

\item Diseño del Sistema:

En este capítulo se recoge la arquitectura general del sistema de información, el diseño de la interfaz de usuario, el diseño físico de datos y el diseño de componentes software.

\item Implementación del sistema:
 
En este capítulo se describe todos los aspectos relacionados con la implementación del sistema en código, haciendo uso de un determinado entorno tecnológico.

\item Pruebas del sistema:

En este capítulo se documentan los diferentes tipos de pruebas, con su resultados y conclusiones, que se han llevado a cabo.

\item Manual de usuario:

 En este capitulo se describe el documento destinado a dar asistencia y formación a las personas que utilizan nuestro sistema.
\item Manual de instalación:

 En este capitulo de detallarán los pasos para poder poner el sistema correctamente en funcionamiento.
\item Conclusiones:

 En este capitulo de detallarán las conclusiones del proyecto realizado.
 \end{enumerate}
 