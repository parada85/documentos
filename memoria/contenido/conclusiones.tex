% ------------------------------------------------------------------------------
% Este fichero es parte de la plantilla LaTeX para la realización de Proyectos
% Final de Grado, protegido bajo los términos de la licencia GFDL.
% Para más información, la licencia completa viene incluida en el
% fichero fdl-1.3.tex

% Copyright (C) 2012 SPI-FM. Universidad de Cádiz
% ------------------------------------------------------------------------------

En este último capítulo detallaremos las lecciones aprendidas tras el desarrollo del presente proyecto e identificaremos las posibles oportunidades de mejora sobre el software desarrollado.

\section{Objetivos}

Además de cumplir con todos los objetivos marcados por el cliente y descritos en la introducción, hemos cumplido con objetivos personales como mejorar la formación, familiarización de nuevas tecnologías no aprendidas en el transcurso de la carrera y experiencia en el desarrollo real de un sistema software de gestión. 

\section{Lecciones aprendidas}

La realización de este proyecto es poner en práctica todos los conocimientos aprendidos durante toda la carrera pero se han tenido que aprender a usar herramientas que eran necesarias para cumplir los objetivos del software.\\

Para la realización de la documentación se ha puesto en práctica todo lo aprendido en ingeniería del software haciendo el análisis y diseño de una aplicación completa y real además del uso de herramientas para la realización de diagramas.\\

Para la implementación se han aprendido multitud de lenguajes de programación los cuales no los había usado anteriormente y pueden ser muy valiosos de cara al futuro en el ámbito laboral. Estos lenguajes son: jquery, php, css, html, xml.

También he aprendido a trabajar en solitario, poniéndome fechas, leyendo manuales y participando en foros. A todo este trabajo en solitario se le unía la dificultad del idioma, por lo que he tenido que aprender obligatoriamente mucho inglés técnico.

\section{Trabajo futuro}

En las últimas reuniones con el cliente, se plantearon algunas mejoras que podrían completar el software. Algunas de estas mejoras son las siguientes:

\begin {itemize}
\item Registro de citas por internet: Se planteó la posibilidad de que el cliente pudiese coger citas por internet, ya que de esta manera no se tendría que desplazar para coger una cita ni llamar por teléfono dando mucha comodidad al cliente.
\item Realización de una aplicación móvil: Se planteó la posibilidad de que el administrador pudiese instalar en su smartphone un programa para que le avisase en tiempo real de información de las operaciones realizadas en el sistema.
\end{itemize}