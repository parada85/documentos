% ------------------------------------------------------------------------------
% Este fichero es parte de la plantilla LaTeX para la realización de Proyectos
% Final de Grado, protegido bajo los términos de la licencia GFDL.
% Para más información, la licencia completa viene incluida en el
% fichero fdl-1.3.tex

% Copyright (C) 2012 SPI-FM. Universidad de Cádiz
% ------------------------------------------------------------------------------

\section{Introducción}
En esta sección describiremos los pasos necesarios para poner al sistema en correcto funcionamiento. 

\section{Requisitos previos}
Los requisitos previos para poder instalar el software son los siguientes:
\begin{itemize}
\item La versión de PHP debe ser 5.3.3 o superior.
\item Es necesario un servidor web como Apache.
\item Es necesario habilitar JSON.
\item Es necesario tener habilitado el ctype.
\item El PHP.ini tiene que tener bien configurado el valor del date.timezone.
\item Es necesaria, al menos, una base de datos MySQL.
\end{itemize}

\section{Procedimientos de instalación}
Distinguiremos dos procedimientos de instalación dependiendo si el servidor permite que nos conectemos mediante SSH. Con cualquiera de estas dos opciones se creará un usuario administrador de prueba con los siguientes datos.
\begin{itemize}
\item \textbf{Nombre usuario}: empleado0
\item \textbf{Contraseña}: empleado0
\end{itemize}

\subsection{Sin conexión SSH}
Para poner en funcionamiento la aplicación en un servidor sin conexión SSH copiaremos con la ayuda de cualquier cliente FTP la carpeta del proyecto a la carpeta principal del servidor la cual suele llamarse Public\_html. Es recomendable transferir la aplicación comprimida para luego descomprimirla en el servidor.\\
En este momento, como ya tenemos la aplicación instalada, podremos comprobar que los requisitos previamente citados se cumplen. Para ello, podremos entrar en la siguiente URL:
\begin{itemize}
\item proyecto/web/config.php
\end{itemize}
A continuación, tendremos que modificar en el archivo parameters.ini las configuraciones del servidor MySQL y del servidor de email. Finalmente, se facilita un archivo database.sql el cual podremos importar a nuestro servidor MySQL utilizando en el servidor PhpMyAdmin.

\subsection{Con conexion SSH}

Si en el servidor disponemos de conexión SSH podremos descargar la aplicación por GIT, utilizando para ello el siguiente comando:
\begin{itemize}
\item GIT clone https://github.com/parada85/optinet.git
\end{itemize}
En este momento, como ya tenemos Symfony2 instalado, podremos comprobar que los requisitos previamente citados se cumplen. Para ello, podremos ejecutar el siguiente comando desde la linea de órdenes:
\begin{itemize}
\item php app/check.php
\end{itemize}

A continuación, tendremos que modificar en el archivo parameters.ini las configuraciones del servidor MySQL y del servidor de email. El siguiente paso es ejecutar estas órdenes desde la linea de comandos en la carpeta del proyecto:
\begin{itemize}
\item app/console doctrine:database:create 
\item app/console doctrine:schema:create
\item app/console doctrine:fixtures:load
\end{itemize}

\section{Procedimientos de operación y nivel de servicio}

En el servidor sería recomendable que configurásemos una tarea programada para que cada cierto tiempo se hiciera un backup de la base de datos para asegurarnos que de ninguna manera perdamos los datos de la base de datos. Esto se hace haciendo uso del comando crontab, por ejemplo:

\begin{itemize}
\item 30 6 1 * * mysqldump -h servidor\_mysql -u user -ppass --all-databases > copia.sql
\end{itemize}

Esto realizaría una copia al principio de cada mes.
