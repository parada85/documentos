% ------------------------------------------------------------------------------
% Este fichero es parte de la plantilla LaTeX para la realización de Proyectos
% Final de Grado, protegido bajo los términos de la licencia GFDL.
% Para más información, la licencia completa viene incluida en el
% fichero fdl-1.3.tex

% Copyright (C) 2012 SPI-FM. Universidad de Cádiz
% ------------------------------------------------------------------------------


\chapter*{\rlap{Información sobre Licencia}}
\phantomsection  % so hyperref creates bookmarks
\addcontentsline{toc}{chapter}{Información sobre Licencia}
%\label{label_fdl}

 \begin{center}

Copyright (C) 1989, 1991 Free Software Foundation, Inc.

675 Mass Ave, Cambridge, MA 02139, EEUU

Se permite la copia y distribución de copias literales de este documento, pero no se permite su modificación.


\end{center}

Preámbulo

Las licencias que cubren la mayor parte del software están diseñadas para quitarle a usted la libertad de compartirlo y modificarlo. Por el contrario, la Licencia Pública General de GNU pretende garantizarle la libertad de compartir y modificar software libre, para asegurar que el software es libre para todos sus usuarios. Esta Licencia Pública General se aplica a la mayor parte del software del la Free Software Foundation y a cualquier otro programa si sus autores se comprometen a utilizarla. (Existe otro software de la Free Software Foundation que está cubierto por la Licencia Pública General de GNU para Bibliotecas). Si quiere, también puede aplicarla a sus propios programas.
Cuando hablamos de software libre, estamos refiriéndonos a libertad, no a precio. Nuestras Licencias Públicas Generales están diseñadas para asegurarnos de que tenga la libertad de distribuir copias de software libre (y cobrar por ese servicio si quiere), de que reciba el código fuente o que pueda conseguirlo si lo quiere, de que pueda modificar el software o usar fragmentos de él en nuevos programas libres, y de que sepa que puede hacer todas estas cosas.

Para proteger sus derechos necesitamos algunas restricciones que prohiban a cualquiera negarle a usted estos derechos o pedirle que renuncie a ellos. Estas restricciones se traducen en ciertas obligaciones que le afectan si distribuye copias del software, o si lo modifica.

Por ejemplo, si distribuye copias de uno de estos programas, sea gratuitamente, o a cambio de una contraprestación, debe dar a los receptores todos los derechos que tiene. Debe asegurarse de que ellos también reciben, o pueden conseguir, el código fuente. Y debe mostrarles estas condiciones de forma que conozcan sus derechos.

Protegemos sus derechos con la combinación de dos medidas:

Ponemos el software bajo copyright y
le ofrecemos esta licencia, que le da permiso legal para copiar, distribuir y/o modificar el software.
También, para la protección de cada autor y la nuestra propia, queremos asegurarnos de que todo el mundo comprende que no se proporciona ninguna garantía para este software libre. Si el software se modifica por cualquiera y éste a su vez lo distribuye, queremos que sus receptores sepan que lo que tienen no es el original, de forma que cualquier problema introducido por otros no afecte a la reputación de los autores originales.
Por último, cualquier programa libre está constantemente amenazado por patentes sobre el software. Queremos evitar el peligro de que los redistribuidores de un programa libre obtengan patentes por su cuenta, convirtiendo de facto el programa en propietario. Para evitar esto, hemos dejado claro que cualquier patente debe ser pedida para el uso libre de cualquiera, o no ser pedida.

Los términos exactos y las condiciones para la copia, distribución y modificación se exponen a continuación.

Términos y condiciones para la copia, distribución y modificación

Esta Licencia se aplica a cualquier programa u otro tipo de trabajo que contenga una nota colocada por el tenedor del copyright diciendo que puede ser distribuido bajo los términos de esta Licencia Pública General. En adelante, «Programa» se referirá a cualquier programa o trabajo que cumpla esa condición y «trabajo basado en el Programa» se referirá bien al Programa o a cualquier trabajo derivado de él según la ley de copyright. Esto es, un trabajo que contenga el programa o una proción de él, bien en forma literal o con modificaciones y/o traducido en otro lenguaje. Por lo tanto, la traducción está incluida sin limitaciones en el término «modificación». Cada concesionario (licenciatario) será denominado «usted».
Cualquier otra actividad que no sea la copia, distribución o modificación no está cubierta por esta Licencia, está fuera de su ámbito. El acto de ejecutar el Programa no está restringido, y los resultados del Programa están cubiertos únicamente si sus contenidos constituyen un trabajo basado en el Programa, independientemente de haberlo producido mediante la ejecución del programa. El que esto se cumpla, depende de lo que haga el programa.

Usted puede copiar y distribuir copias literales del código fuente del Programa, según lo has recibido, en cualquier medio, supuesto que de forma adecuada y bien visible publique en cada copia un anuncio de copyright adecuado y un repudio de garantía, mantenga intactos todos los anuncios que se refieran a esta Licencia y a la ausencia de garantía, y proporcione a cualquier otro receptor del programa una copia de esta Licencia junto con el Programa.
Puede cobrar un precio por el acto físico de transferir una copia, y puede, según su libre albedrío, ofrecer garantía a cambio de unos honorarios.

Puede modificar su copia o copias del Programa o de cualquier porción de él, formando de esta manera un trabajo basado en el Programa, y copiar y distribuir esa modificación o trabajo bajo los términos del apartado 1, antedicho, supuesto que además cumpla las siguientes condiciones:
Debe hacer que los ficheros modificados lleven anuncios prominentes indicando que los ha cambiado y la fecha de cualquier cambio.
Debe hacer que cualquier trabajo que distribuya o publique y que en todo o en parte contenga o sea derivado del Programa o de cualquier parte de él sea licenciada como un todo, sin carga alguna, a todas las terceras partes y bajo los términos de esta Licencia.
Si el programa modificado lee normalmente órdenes interactivamente cuando es ejecutado, debe hacer que, cuando comience su ejecución para ese uso interactivo de la forma más habitual, muestre o escriba un mensaje que incluya un anuncio de copyright y un anuncio de que no se ofrece ninguna garantía (o por el contrario que sí se ofrece garantía) y que los usuarios pueden redistribuir el programa bajo estas condiciones, e indicando al usuario cómo ver una copia de esta licencia. (Excepción: si el propio programa es interactivo pero normalmente no muestra ese anuncio, no se requiere que su trabajo basado en el Programa muestre ningún anuncio).
Estos requisitos se aplican al trabajo modificado como un todo. Si partes identificables de ese trabajo no son derivadas del Programa, y pueden, razonablemente, ser consideradas trabajos independientes y separados por ellos mismos, entonces esta Licencia y sus términos no se aplican a esas partes cuando sean distribuidas como trabajos separados. Pero cuando distribuya esas mismas secciones como partes de un todo que es un trabajo basado en el Programa, la distribución del todo debe ser según los términos de esta licencia, cuyos permisos para otros licenciatarios se extienden al todo completo, y por lo tanto a todas y cada una de sus partes, con independencia de quién la escribió.
Por lo tanto, no es la intención de este apartado reclamar derechos o desafiar sus derechos sobre trabajos escritos totalmente por usted mismo. El intento es ejercer el derecho a controlar la distribución de trabajos derivados o colectivos basados en el Programa.

Además, el simple hecho de reunir un trabajo no basado en el Programa con el Programa (o con un trabajo basado en el Programa) en un volumen de almacenamiento o en un medio de distribución no hace que dicho trabajo entre dentro del ámbito cubierto por esta Licencia.

Puede copiar y distribuir el Programa (o un trabajo basado en él, según se especifica en el apartado 2, como código objeto o en formato ejecutable según los términos de los apartados 1 y 2, supuesto que además cumpla una de las siguientes condiciones:
Acompañarlo con el código fuente completo correspondiente, en formato electrónico, que debe ser distribuido según se especifica en los apartados 1 y 2 de esta Licencia en un medio habitualmente utilizado para el intercambio de programas, o
Acompañarlo con una oferta por escrito, válida durante al menos tres años, de proporcionar a cualquier tercera parte una copia completa en formato electrónico del código fuente correspondiente, a un coste no mayor que el de realizar físicamente la distribución del fuente, que será distribuido bajo las condiciones descritas en los apartados 1 y 2 anteriores, en un medio habitualmente utilizado para el intercambio de programas, o
Acompañarlo con la información que recibiste ofreciendo distribuir el código fuente correspondiente. (Esta opción se permite sólo para distribución no comercial y sólo si usted recibió el programa como código objeto o en formato ejecutable con tal oferta, de acuerdo con el apartado b anterior).
Por código fuente de un trabajo se entiende la forma preferida del trabajo cuando se le hacen modificaciones. Para un trabajo ejecutable, se entiende por código fuente completo todo el código fuente para todos los módulos que contiene, más cualquier fichero asociado de definición de interfaces, más los guiones utilizados para controlar la compilación e instalación del ejecutable. Como excepción especial el código fuente distribuido no necesita incluir nada que sea distribuido normalmente (bien como fuente, bien en forma binaria) con los componentes principales (compilador, kernel y similares) del sistema operativo en el cual funciona el ejecutable, a no ser que el propio componente acompañe al ejecutable.
Si la distribución del ejecutable o del código objeto se hace mediante la oferta acceso para copiarlo de un cierto lugar, entonces se considera la oferta de acceso para copiar el código fuente del mismo lugar como distribución del código fuente, incluso aunque terceras partes no estén forzadas a copiar el fuente junto con el código objeto.

No puede copiar, modificar, sublicenciar o distribuir el Programa excepto como prevé expresamente esta Licencia. Cualquier intento de copiar, modificar sublicenciar o distribuir el Programa de otra forma es inválida, y hará que cesen automáticamente los derechos que te proporciona esta Licencia. En cualquier caso, las partes que hayan recibido copias o derechos de usted bajo esta Licencia no cesarán en sus derechos mientras esas partes continúen cumpliéndola.
No está obligado a aceptar esta licencia, ya que no la ha firmado. Sin embargo, no hay hada más que le proporcione permiso para modificar o distribuir el Programa o sus trabajos derivados. Estas acciones están prohibidas por la ley si no acepta esta Licencia. Por lo tanto, si modifica o distribuye el Programa (o cualquier trabajo basado en el Programa), está indicando que acepta esta Licencia para poder hacerlo, y todos sus términos y condiciones para copiar, distribuir o modificar el Programa o trabajos basados en él.
Cada vez que redistribuya el Programa (o cualquier trabajo basado en el Programa), el receptor recibe automáticamente una licencia del licenciatario original para copiar, distribuir o modificar el Programa, de forma sujeta a estos términos y condiciones. No puede imponer al receptor ninguna restricción más sobre el ejercicio de los derechos aquí garantizados. No es usted responsable de hacer cumplir esta licencia por terceras partes.
Si como consecuencia de una resolución judicial o de una alegación de infracción de patente o por cualquier otra razón (no limitada a asuntos relacionados con patentes) se le imponen condiciones (ya sea por mandato judicial, por acuerdo o por cualquier otra causa) que contradigan las condiciones de esta Licencia, ello no le exime de cumplir las condiciones de esta Licencia. Si no puede realizar distribuciones de forma que se satisfagan simultáneamente sus obligaciones bajo esta licencia y cualquier otra obligación pertinente entonces, como consecuencia, no puede distribuir el Programa de ninguna forma. Por ejemplo, si una patente no permite la redistribución libre de derechos de autor del Programa por parte de todos aquellos que reciban copias directa o indirectamente a través de usted, entonces la única forma en que podría satisfacer tanto esa condición como esta Licencia sería evitar completamente la distribución del Programa.
Si cualquier porción de este apartado se considera inválida o imposible de cumplir bajo cualquier circunstancia particular ha de cumplirse el resto y la sección por entero ha de cumplirse en cualquier otra circunstancia.

No es el propósito de este apartado inducirle a infringir ninguna reivindicación de patente ni de ningún otro derecho de propiedad o impugnar la validez de ninguna de dichas reivindicaciones. Este apartado tiene el único propósito de proteger la integridad del sistema de distribución de software libre, que se realiza mediante prácticas de licencia pública. Mucha gente ha hecho contribuciones generosas a la gran variedad de software distribuido mediante ese sistema con la confianza de que el sistema se aplicará consistentemente. Será el autor/donante quien decida si quiere distribuir software mediante cualquier otro sistema y una licencia no puede imponer esa elección.

Este apartado pretende dejar completamente claro lo que se cree que es una consecuencia del resto de esta Licencia.

Si la distribución y/o uso de el Programa está restringida en ciertos países, bien por patentes o por interfaces bajo copyright, el tenedor del copyright que coloca este Programa bajo esta Licencia puede añadir una limitación explícita de distribución geográfica excluyendo esos países, de forma que la distribución se permita sólo en o entre los países no excluidos de esta manera. En ese caso, esta Licencia incorporará la limitación como si estuviese escrita en el cuerpo de esta Licencia.
La Free Software Foundation puede publicar versiones revisadas y/o nuevas de la Licencia Pública General de tiempo en tiempo. Dichas nuevas versiones serán similares en espíritu a la presente versión, pero pueden ser diferentes en detalles para considerar nuevos problemas o situaciones.
Cada versión recibe un número de versión que la distingue de otras. Si el Programa especifica un número de versión de esta Licencia que se refiere a ella y a «cualquier versión posterior», tienes la opción de seguir los términos y condiciones, bien de esa versión, bien de cualquier versión posterior publicada por la Free Software Foundation. Si el Programa no especifica un número de versión de esta Licencia, puedes escoger cualquier versión publicada por la Free Software Foundation.

Si quiere incorporar partes del Programa en otros programas libres cuyas condiciones de distribución son diferentes, escribe al autor para pedirle permiso. Si el software tiene copyright de la Free Software Foundation, escribe a la Free Software Foundation: algunas veces hacemos excepciones en estos casos. Nuestra decisión estará guiada por el doble objetivo de de preservar la libertad de todos los derivados de nuestro software libre y promover el que se comparta y reutilice el software en general.
AUSENCIA DE GARANTÍA

Como el programa se licencia libre de cargas, no se ofrece ninguna garantía sobre el programa, en todas la extensión permitida por la legislación aplicable. Excepto cuando se indique de otra forma por escrito, los tenedores del copyright y/u otras partes proporcionan el programa «tal cual», sin garantía de ninguna clase, bien expresa o implícita, con inclusión, pero sin limitación a las garantías mercantiles implícitas o a la conveniencia para un propósito particular. Cualquier riesgo referente a la calidad y prestaciones del programa es asumido por usted. Si se probase que el Programa es defectuoso, asume el coste de cualquier servicio, reparación o corrección.
En ningún caso, salvo que lo requiera la legislación aplicable o haya sido acordado por escrito, ningún tenedor del copyright ni ninguna otra parte que modifique y/o redistribuya el Programa según se permite en esta Licencia será responsable ante usted por daños, incluyendo cualquier daño general, especial, incidental o resultante producido por el uso o la imposibilidad de uso del Programa (con inclusión, pero sin limitación a la pérdida de datos o a la generación incorrecta de datos o a pérdidas sufridas por usted o por terceras partes o a un fallo del Programa al funcionar en combinación con cualquier otro programa), incluso si dicho tenedor u otra parte ha sido advertido de la posibilidad de dichos daños.
